% Introduction chapter

\chapter{Introduction}

\label{chp:intro}

\begin{refsection}

%----------------------------------------------------------------------------------------
%	SECTION 1
%----------------------------------------------------------------------------------------

\section{Motivation} \label{sec:motivation}
Our understanding of the climate system is based on computational models that operate on large spatial scales and simulate a complex system of closely related environmental parameters. The evaluation of these models poses a considerable challenge because we need data on continental scale to reconstruct large-scale climatic features, such as atmospheric dynamics or latitudinal temperature gradients. This instrumental data, however, is limited to not even the past two centuries and overlaps highly with the \textit{comfort zone} of the models, i.e. the period where they have both been developed and tested. 

A comparison of models with paleo-environmental proxy records, i.e. records from past climates prior to the systematic measurement of meteorology and climatology, provides therefore the only possibility to evaluate the predictive skill of our models for climates that are very different from today. The Holocene, ranging from 11'700 years ago to present, provides an ideal basis for it because (1) it is recent enough so that boundary conditions and forcings are well known, and (2) paleoenvironmental archives are abundant and dated with enough precision to comprehensively reconstruct climate. Each of these archives represent the regional climate condition in the surrounding environment and, when grouped together into large databases; they allow an informed estimate of the climate state over a large period of time and vast geographic areas.

A direct comparison of climate model output and proxies is however still challenging because even the  climate proxy record, as an indirect measurement of climate, relies on an inverse modelling approach with associated uncertainties that are not always easy to quantify.

Key challenges for large-scale data-model comparisons on past climates are therefore (1) to gather enough climate proxy information from a spatially large area and a variety of climates, and (2) to provide reliable estimates of the uncertainties associated to the indirect measurements.

These challenges will be addressed in this thesis via the development of new software tools that cover flexible data analysis (chapters \ref{chp:empd} and \ref{chp:psyplot}), a tool for data gathering (chapter \ref{chp:straditize}), as well as new predictive methods for large-scale paleoenvironmental modelling (chapters \ref{chp:gridding} and \ref{chp:gwgen}).

All these tools are open-source with a strong emphasis on a proper software development that includes documentation and reproducibility. 

In the following section \ref{sec:intro-paleo}, I will describe the paleo-climate of the current epoch and why this is of interest for future climate predictions. In the subsequent section \ref{sec:intro-software}, I introduce the influence of software development in this paleo-climatic research, and some of the common open-source software development contents. I conclude this chapter by providing an overview on the contents of this thesis in section \ref{sec:intro-software-tools}.

\section{Learning from the past –- Why we study paleo-climates} \label{sec:intro-paleo}

Mankind is facing large infrastructural challenges during this century, such as the loss of biodiversity, an exponentially growing world population and an acceleration of growth and globalization of markets \citep[e.g.][]{CeballosEhrlichBarnoskyEtAl2015, UniNations2019, WorldBank2002}. They all interact with a global climate change that may lead to a new environment none of us ever experienced \citep{CollinsKnuttiArblasterEtAl2013}. Any future global planning has to account highly diverse responses that range from regional to continental scales \citep{ChristensenKrishnaKumarAldrianEtAl2013}. As such the complex climate system will enter a state that is significantly different from everything we had since the beginning of the satellite era, i.e. the beginning of global meteorological data acquisition, and even different from what has been experienced within the last 2'000 years \citep{NeukomSteigerGomezNavarroEtAl2019, NeukomBarbozaErbEtAl2019}.

Our knowledge about this new climate is therefore mainly based on computational \glspl{esm}. They face the challenge of simulating a new climate based on our present knowledge of the interactions between the different compartments Ocean, Land and Atmosphere. The validation of it becomes conceptually difficult because of the aforementioned transition into a warmer world during the next century. We are entering a new state and it is questionable how well our models perform \citep{HargreavesAnnanOhgaitoEtAl2013, MauriDavisCollinsEtAl2014}.

To evaluate the predictive skill, we rely on our knowledge of paleo-climates, i.e. climates before the systematic measurement of temperature, precipitation, etc.. They provide the only opportunity for a large-scale evaluation of \glspl{esm} under conditions very different than today. Paleo-climatic research has therefore been an integral part for climate sciences since the 80s \citep{COHMAPMembers1988, JoussaumeTaylor1995}, particularly in the \glsfirst{pmip} \citep{BraconnotOttoBliesnerHarrisonEtAl2007, BraconnotOttoBliesnerHarrisonEtAl2007a, BraconnotHarrisonKageyamaEtAl2012, KageyamaBraconnotHarrisonEtAl2016, Otto-BliesnerBraconnotHarrisonEtAl2017, JungclausBardBaroniEtAl2017}.

The Holocene interglacial period (11'700 years ago to present)  \citep{WalkerJohnsenRasmussenEtAl2009} is particularly important because it is sufficiently close in time to provide paleo-climate archives and the forcings and boundary conditions are well known \citep{WannerBeerButikoferEtAl2008}. With the end of the Younger Dryas around 11'700 years ago, the Earth experienced a climate warming due to changes in orbital precession and obliquity of the Earth, as well as the disappearing residual ice sheets of the \gls{lgm} \citep{BergerLoutre1991, Peltier2004}. This results in a multitude of large-scale effects in the atmospheric circulation, such as an increasing amplitude and frequency of the El Niño–Southern Oscillation (ENSO) \citep{DondersWagnerCremerVisscher2008}, stronger westerly circulation in winter indicating a more positive AO/NAO over mid-latitudes and the arctic \citep{MauriDavisCollinsEtAl2014, FunderGoosseJepsenEtAl2011} and changes in the polar amplification and a weakening of the latitudinal temperature gradient \citep{DavisBrewer2009}.

Hence, this epoch is of particular interest because the continental setup is comparable to nowadays while still having a climate that is significantly different from present day.

\subsection{Pollen as a climate proxy}  \label{sec:intro-paleo-data}
Before 1850, there is almost no instrumental measurement of temperature. Instead we rely on archives such as lake sediments, glaciers, peat bogs, or speleothems that preserve climate proxies. The latter is a set of variables that are influenced by climate conditions and therefore allow an indirect measurement of climate(-related) parameters at ancient times, e.g. temperature, precipitation or sea-level. 

The most abundant climate proxy, that I will also focus on in the next chapters, are pollen assemblages. It is the  geographically most spread paleo-climate proxy \citep{BirksBirks1980} and has a long history in quantitative paleo-climatologic reconstructions \citep[e.g.][]{Nichols1967, Nichols1969, Bradley1985, Iversen1944}.

The chemically stable polymer sporopollenin allows the pollen grain to be preserved over very long periods of time, in various terrestrial archives such as lakes, wetlands or ocean sediments \citep{FaegriKalandKrzywinski1989, Havinga1967}. Pollen are produced by seed-bearing plants (spermatophytes, \cite{Wodehouse1935}) and as such have a high spatial continuity and prevalence \citep{ChevalierDavisGajewskiEtAlinprep}. Their compositions are strongly influenced by the surrounding climate, although other factors, such as soil compositions or inter-species competition also play an important role. This dependency allows to reconstruct the driving factor, i.e. climate parameters such as winter and summer temperature, or precipitation from the observed pollen data \citep{JugginsBirks2012,Juggins2013,BrewerGuiotBarboni2007, ChevalierDavisGajewskiEtAlinprep}.

This high abundance of pollen led to multiple regional efforts to combine and homogenize fossil pollen data. This makes pollen particularly useful for large-scale data-model intercomparisons. The earliest examples are the \glstarget{epd}{European Pollen Data-base (EPD)} and \gls{napd}  that both started around 1990 and developed a similar structure in order to be compatible \citep{Grimm2008, FyfeBeaulieuBinneyEtAl2009} . This let to the development of other regional pollen databases, such as the \gls{lapd} \citep{FlantuaHooghiemstraGrimmEtAl2015, MarchantAlmeidaBehlingEtAl2002} in 1994 or the \gls{apd} \citep{VincensLezineBuchetEtAl2007} in 1996, and others \citep[see][]{Grimm2008}. These attempts finally let to the development of the Neotoma database \citep{WilliamsGrimmBloisEtAl2018}, a global multiproxy database that incorporates many of the regional pollen databases.

The use of pollen for paleo-climate reconstruction has a long academic tradition in geology \citep{Bradley1985} and provides the source of large-scale paleo-climatic reconstructions in number of different studies \citep[and more]{MauriDavisCollinsEtAl2015, DavisBrewerStevensonEtAl2003, MarsicekShumanBartleinEtAl2018, FischerJungclaus2011}. Such a reconstruction, however, has multiple uncertainties that are often difficult to quantify and to consider (see chapter \ref{chp:gridding}). Key challenges for a data-model comparison are dating uncertainties, influences of seasonality on the proxy (e.g. whether it represents summer, winter or annual temperature) and quality and temporal resolutions of the record. Another challenge is the proper handling of uncertainties related to the inverse modelling approach \citep[e.g.][]{GuiotVernal2011, TelfordBirks2009, TelfordBirks2005}, and the spatial coverage of the proxy (see chapter \ref{chp:straditize}).

%----------------------------------------------------------------------------------------
%	SECTION 2
%----------------------------------------------------------------------------------------

\section{Software for Paleoclimatology} \label{sec:intro-software}

The usage of software is crucial for the quantitative reconstruction of Earth's Climate. Paleoclimate research is facing an information overload problem and requires innovative methods in the realm of visual analytics, i.e. the interplay between automated analysis techniques and interactive visualization \citep{KeimAndrienkoFeketeEtAl2008, Nocke2014}. As such, a visual representation of the paleoclimate reconstruction has been essential for both, proxies \citep{Nichols1967, Bradley1985, Grimm1988} and models \citep{Phillips1956, RautenhausBoettingerSiemenEtAl2018, NockeSterzelBoettingerEtAl2008, Nocke2014, BoettingerRoeber2019}, although the visualization methods significantly differ due to the differences in data size and data heterogeneity.

The second important aspect for software and paleoclimate is the distribution of data to make it accessible to other researchers, the community and policy makers, which is commonly established through online accessible data archives and recently also through map-based web interfaces \citep{WilliamsGrimmBloisEtAl2018, BollietBrockmannMassonDelmotteEtAl2016}.

The following sections provide an overview on the different techniques used by palynologists to visualize and distribute their data and concludes with an introduction into Open-Source Software Development, which forms the basis of all the software solutions that are presented later in this thesis.

\subsection{Sofware for Proxy Data Analysis, Visualization and Distribution} \label{sec:intro-software-data}

Due to the nature of stratigraphic data, proxies, especially pollen assemblages, are often treated as a collection of multiple time-series (one-dimensional arrays). The size of one dataset is generally small (in the range of kB) and can be treated as plain text files. Traditionally, numerical and statistical analysis are separated from the visualization.

In palynology, standard analytical tools are Microsoft Excel\footnote{\url{https://products.office.com/en/excel}} and the R software for statistical computing \citep{RCT2019}. The latter also involves multiple packages for paleoclimatic reconstruction, such as \texttt{rioja} \citep{Juggins2017} and \texttt{analogue} \citep{SimpsonOksanen2019, Simpson2007}, and bayesian methods exists in a variety of programming languages \citep[e.g.][]{NolanTiptonBoothEtAl2019, Tipton2019, HaslettWhileyBhattacharyaEtAl2006,ParnellSweeneyDoanEtAl2015, HolmstroemIlvonenSeppaeEtAl2015}. Alternatively, desktop applications exist, such as Polygon\footnote{\url{http://polsystems.rits-paleo.com}} by \cite{NakagawaTarasovNishidaEtAl2002} or the CREST software presented by \cite{ChevalierCheddadiChase2014} and \cite{Chevalier2019}.

It is a long-standing tradition to visualize stratigraphic data, and especially pollen data, in form of a stratigraphic (pollen) diagram \citep{Bradley1985, Grimm1988}. Especially during the 20th century, when it was not yet common to distribute data alongside a peer-reviewed publication, pollen diagrams have been the only possibility to publish the entire dataset (see also chapter \ref{chp:straditize}). The generation of these diagrams is usually based on desktop applications such as C2 \citep{Juggins2007} or Tilia\footnote{\url{https://www.tiliait.com/}} \citep{Grimm1988, Grimm1991}. A more recent implementation into the psyplot framework \citep[chapter \ref{chp:psyplot}]{Sommer2017} is also provided with the psy-strat plugin\footnote{\url{https://psy-strat.readthedocs.io}} \citep{Sommer2019}.

Raw pollen data is at present made available through web archives, such as PANGAEA\footnote{\url{https://pangaea.de/}} or the \gls{ncdc} by the \glsfirst{noaa}\footnote{\url{https://www.ncdc.noaa.gov/data-access/paleoclimatology-data}}. Collections of data, such as regional pollen databases or project specific collections \citep[e.g.][]{WhitmoreGajewskiSawadaEtAl2005, DavisZanonCollinsEtAl2013} are usually published in one of the above-mentioned archives or associated with a publication. A different approach has been taken by \cite{BollietBrockmannMassonDelmotteEtAl2016} who developed a small web application as an interface into the data collection, the \textit{ClimateProxiesFinder} \citep[chapter \ref{chp:empd}]{Brockmann2016}.

Outstanding compared to the previous data interfaces is the new infrastructure for the Neotoma database \citep{WilliamsGrimmBloisEtAl2018}. It consists of the map-based web interface, the Neotoma Explorer\footnote{\url{https://apps.neotomadb.org/Explorer}}, a RESTful api\footnote{\url{https://api.neotomadb.org}} that allows an interaction with other web services, the neotoma R package \citep{GoringDawsonSimpsonEtAl2015} and an interface into the Tilia software for stratigraphic and map-based visualizations \citep{WilliamsGrimmBloisEtAl2018}. This rich functionality is, however, bound to the structure of Neotoma and as such, different from the Javascript-based approach developed in chapter \ref{chp:empd} because it cannot easily be transferred to other projects.


\subsection{Methods and Workflows in Open-Source Software Development} \label{sec:intro-software-tools}

The importance and necessity of software for visualization and data analysis led to the development of the software packages I present in this thesis. Most of them are written in the programming language Python \citep{PerezGrangerHunter2011}, on the one hand due to my personal preference, but mainly due to the recent developments in out-of-core computing with the establishment of xarray and dask \citep{HoyerHamman2017, DDT2016, Rocklin2015}. Another important reason, especially for psyplot (chapter \ref{chp:psyplot}) and straditize (chapter \ref{chp:straditize}) was the availability of a highly flexible and stable package for graphical user interfaces, PyQt \footnote{\url{http://pyqt.sourceforge.net/Docs/PyQt5/installation.html}}, and the comparably simple possibility to implement an in-process python console into the PyQ5 application\footnote{\url{https://qtconsole.readthedocs.org}} that allows to handle the software functionalities both, from the command line and from the GUI.

The tools that I present in the following chapter are all available as open-source software packages. But modern \gls{foss} development is not only about making the source code available, but rather about providing a sustainable and maintainable package that allows continuous and transparent development under the aspect of rapidly evolving environment. In the following sections, I will introduce the most important \gls{foss} development concepts \citep[e.g.][]{StoddenMiguez2014, Shaw2018} and the necessary vocabulary. These concepts are used by many of the well-established software packages, such as matplotlib \citep{Hunter2007}, numpy \citep{Oliphant2006}, and scipy \citep{JonesOliphantPetersonEtAl2001}.

\subsubsection{Version Control} \label{sec:intro-software-github}
Version control systems record changes to a file and enables the user to roll-back to previous versions of it. The usage of a such a system is inevitable for sustainable \gls{foss} packages. It enables contributions by other \gls{foss} developers and the usage through external packages.

The packages I present in the following chapters are hosted on Github\footnote{The packages are available at \url{https://github.com/Chilipp}. Other potential platforms for version control are sourceforge (\url{https://sourceforge.net}) and Bitbucket (\url{https://bitbucket.org})}, a freely available web platform for hosting projects that are managed with git \citep{ChaconStraubPGC2019}.

Version control with git has a specific terminology (see also chapter \ref{chp:empd}). Central aspects are \textit{repositories} (project folders), \textit{commits} (change of the project files), \textit{issues} (bug reports), \textit{branches} and \textit{forks} (copies of the (main) project), and \textit{pull requests} (contributions to a project). The following list explains this vocabulary in a bit more detail because the terminology is used in several parts of this thesis, particular in chapter \ref{chp:empd} and \ref{chp:psyplot}. A more complete list is provided in \cite{Github2019}.

\begin{description}
	\item[Repositories] are the most basic elements of git and Github. It can be compared to a folder that contains all the necessary files associated with a project (e.g. the source code and documentation of a software package). It also contains all the different versions (revisions) of the project files.
	\item[Commits] or revisions track the changes in the repository. Each commit is a change to a specific file (or a set of files) that is associated with a unique ID and a message of the author that describes the changes.
	\item[Issues] are suggested improvements, bug reports or any other question to the repository. Every issue has an associated discussion page for the communication between repository owners and the users.
	\item[Branches] are parallel versions of a repository. Often one incorporates new developments into a separate branch that does not affect the main version of the repository (the \textit{master} branch) and merge the two versions when the new developments are fully implemented.
	\item[Forks] are copies of repositories. When someone wants to contribute to a software package (repository) that does not belong to him, he can \textit{fork} (copy) it, implement the changes, and then create a \textit{pull request} to contribute to the official version. Different from a branch, that is a (modified) copy of another branch, forks are copies of the entire repository, i.e. all existing branches.
	\item[Pull request] are the proposed changes to a repository. One can create a fork of the repository from someone else, implement changes in this fork and then create a pull request to merge it into the original repository. Every pull request has an associated discussion page that allows the repository owner to moderate and discuss the suggested changes.
	\item[Webhooks] are general methods for web development. Github can trigger a hook to inform a different web service (such as a \gls{ci} service, see next section)) that a repository has changed or that someone contributed in a discussion. In chapter \ref{chp:empd} we use Github webhooks for an automated administration of a repository.
\end{description}


\subsubsection{Automated Tests, Test Coverage and Continuous Integration} \label{sec:intro-software-ci}
The most important aspect for \gls{foss} development, especially considering the rapid evolution of this area, is the existence of automated tests. One distinguishes unit tests (tests designed to cover one specific routine) and integration tests (tests of one or more routines within the framework) \citep{Shaw2018}. The boundary between the two tests is rather vague and the decision about what is used highly depends on the structure of the software that is supposed to be tested. For complex frameworks (such as psyplot or straditize), integration tests are needed to ensure the operability within the framework. Other more simple software packages, (such as docrep or model-organization \citep{Sommer2018a, Sommer2018b}) go well with unit tests only.

Another good standard for such a test suite is to use an automated test discovery tool (e.g. the Python unittest package \citep{PSF2019} or pytest \citep{KrekelOliveiraPfannschmidtEtAl2004}) that also reports the test coverage (i.e. the fraction of the code that is tested by the test suite). These functionalities are then implemented on a \gls{ci} service, such as Travis CI\footnote{\url{https://travis-ci.org/}}, Appveyor\footnote{\url{https://appveyor.com}} or CircleCi\footnote{\url{https://circleci.com/}} that are integrated into the Github repository (section \ref{sec:intro-software-github}). Every commit to the Github repository, or any new pull requests then triggers the tests. This transparently allows to ensure the operability of the software, and the test coverage report ensures that the newly implemented functionality is properly tested. A software development concept that is entirely built on this idea is the test-driven development. Within this framework, new features are implemented by starting with the test that should be fulfilled by the new feature and then improving the software until this test pass \citep{Beck2002}.


\subsubsection{Automated Documentation} \label{sec:intro-software-docs}
Documentation is the key aspect of a sustainable software and much of the geo-scientific code has a lack of proper documentation (based on personal experience). For the software in this thesis, four different levels of the documentation play an important role:

\begin{description}
	\item[The \gls{api} documentation] is meant to \\document the major parts of the software code that is subject to be used by external scripts or packages. It is usually implemented in the code and documents the essential subroutines and methods of the software.
	\item[The \gls{gui} documentation] provides help for the most high-level functionality for the software. The \gls{gui} is a user interface into the software through graphical elements (such as buttons, checkboxes, etc.). Unlike the \gls{api} documentation, it should not require knowledge about programming.
	\item[The contributing and/or developers guide] is targeting other software developers that might want to contribute to the software package. This document states how other software developers should contribute to the software and introduces the central structural aspects and frameworks of the software.
	\item[The manual] (or also commonly referred to as \textit{the} documentation) is the document that contains all necessary information for the software, such as installation instructions, tutorials, examples, etc.. It often includes some (or multiple) of the above parts.
\end{description}

The documentations for the software in this thesis have been automatically generated with Sphinx, a Python tool to generate documentations in various different formats (such as HTML, PDF, etc.)  \citep{PerezGrangerHunter2011, Hasecke2019}. It is also implemented as a webhook into the Github repository (see section \ref{sec:intro-software-github}) to automatically generate an up-to-date documentation of the software for each commit to the Github repository. This provides an additional automated test for the software, and especially its high-level-interface, in addition to the automated test suite described in the previous section. Most of the manuals for the software packages in this thesis are hosted and build online with the free services offered by \href{https://readthedocs.org/}{readthedocs.org}.


\subsubsection{Distribution through package managers and virtual environments} \label{sec:intro-software-conda}

\gls{foss} software is meant to be extensible and to build upon other \gls{foss} packages. This requires an accurate and transparent handling of its dependencies and requirements which is usually provided through the so-called packaging of the software \citep[e.g.][]{Torborg2016}. There exists a variety of package managers and the choice most often depends on the framework of the software.

The software in this thesis is mainly distributed via two systems. The first one is python's own package manager \textit{pip} which is based on the packages uploaded to \href{https://pypi.org/}{pypi.org}. The second one, which got increasing importance during the recent past, is the open-source Anaconda Distribution\footnote{\url{https://www.anaconda.com}}. Both work on multiple operating systems (Windows, Linux and Mac OS), but the Anaconda Distribution contains also non-python packages (e.g. written in C or C++) that multiple Python packages rely on; and it contains a rich suite of R packages.

One step further, compared to package managers, are the distribution of virtual environments. These systems do not only provide the software, but also a full operating system and the installed dependencies. A popular platform (used also for the \gls{empd} database) is provided through so-called Docker containers\footnote{\url{https://www.docker.com}}. Compared to package managers, this system has the advantage of simplifying the installation procedure for the user because he only has to download the corresponding docker image. The docker image itself then runs independent of the local file system in a separate isolated mode.


%----------------------------------------------------------------------------------------
%	SECTION 2
%----------------------------------------------------------------------------------------

\section{Challenges tackled in this thesis} \label{sec:intro-thesis-overview}

In part \ref{part:software} of this thesis I present several new software tools that tackle the data analysis, data gathering and data distribution aspects described in the previous section \ref{sec:intro-software}. 

Chapter \ref{chp:empd} in this first part describes new tools for the data analysis and distribution of pollen data on a large continental scale. In this chapter I present the new infrastructural tools I developed for the sustainable management of the community-driven \glsfirst{empd}. These tools consist of a flexible and lightweight map-based web interface into the data, the \acrshort{empd}-viewer, and a webserver for an automated administration of the database. Within this chapter, I also present another use case for the map-viewer that is adapted to a large northern-hemispheric database of fossil and modern pollen records.

The second chapter in this part, chapter \ref{chp:straditize}, describes the new \textit{straditize} software that addresses the problem of gathering proxy data that has been collected during the pre-digital area. This software is a semi-automated digitization package for stratigraphic diagrams, and particularly pollen diagrams that we use to fill gaps in our database in data-poor regions.

I conclude the first part with the presentation of the generic visualization framework psyplot in chapter \ref{chp:psyplot}. It is a suite of python packages that are designed for an interactive visual analysis of data, both from a \gls{gui} and the command line. This software is the base infrastructure for many of the tools described in the other chapters. It has a very general scope is not limited to paleoclimate analysis.

In the \hyperref[part:models]{second part} I present two new models that leverage site-based observations (or paleo climate reconstruction) onto a continental, or even global scale. The first model in chapter \ref{chp:gridding} presents the very recent \textit{pyleogrid} package that extends the methodology of \citep{MauriDavisCollinsEtAl2015}. The ensemble method I present in this study provides a spatio-temporal gridding of side-based proxy-climate estimates under the consideration of both, their dating and reconstruction uncertainties. The outcome of this model can be conveniently used for data-model intercomparisons because it contains reliable estimates of the methodological uncertainties.

Finally, I describe the weather generator \textit{GWGEN} in chapter \ref{chp:gwgen}, a statistical model that uses modern relationships in observational meteorological data to inform large-scale paleo vegetation models with temporally downscaled temperature, precipitation, cloud cover and wind speed records. This weather generator has been parameterized with more than 50~million daily weather observation to be applicable on the entire globe.

This thesis finishes with the conclusions in chapter \ref{chp:conclusions} where I summarize the new tools from this thesis and provide an outlook for the further development of the methods. Three of them are already published in peer-reviewed journals by the time that this thesis has been submitted. These software packages are psyplot \citep{Sommer2017}, GWGEN \citep{SommerKaplan2017b} and straditize \citep{SommerRechChevalierEtAl2019}. In appendix \ref{chp:publications} I provide a list of all the peer-reviewed publications I have been involved as main or co-author during my thesis and that have been accepted by the time that this thesis has been submitted.

\printbibliography[heading=subbibintoc]

\end{refsection}
