% Conclusions chapter

\chapter{Conclusions} 

\label{chp:conclusions}

\begin{refsection}

Paleoclimatological large-scale reconstructions allow an independent evaluation of global climate models skill to simulate climates outside the range of modern climate variability. In the previous chapters of this thesis I described several new open-source tools that can be used to leverage the single site-based proxy-climate reconstruction onto continental, or even global scale, by using a combination of thousands of different records. 

For the \gls{empd} \citep{DavisChevalierSommerEtAlinprep} I developed a web-framework to communicate and manage the community-driven database in a transparent and sustainable way (chapter \ref{chp:empd}). The framework consists of an interactive web-based interface into the data and an automated administration webapp. The entire framework is based on the free webservices provided by the version control platform Github and as such allow to trace back every change to the database and provides a variety of tools to manage new contributions and/or changes to the database. This methodology assures stable and intuitive access to the database, independent of the available funding, and contributors or maintainers. One can think of many further potential applications for this framework that can be applied to any regional pollen (or in general, proxy) database. The \gls{empd} is only one example, other potential use cases are the \gls{lapd} \citep{FlantuaHooghiemstraGrimmEtAl2015}, or the \gls{apd} \citep{VincensLezineBuchetEtAl2007}. The method can also be applied to communicate a study-specific collection of proxy sites and use already implemented analysis and visualization tools for the data, or add new methods specific to the scope of the study. The future plans with this project therefore include a further generalization of the methods, particularly the visualization methods of the \gls{empd}- and POLNET-viewer (section \ref{sec:polnet-viewer}), to make it widely applicable. The integration with Github allows an easy way to share the source, and to host the interactive interface on the same platform without any costs.

The next tool I presented is the stratigraphic digitization software straditize in chapter \ref{chp:straditize} \citep{SommerRechChevalierEtAl2019}. This package transforms stratigraphic diagrams, i.e. diagrams where the analysis of samples are plotted against a common y-axis, usually representing age or depth. The potential applications for this software are numerous because of the existence of hundreds of pollen datasets (and more) that are only available as pollen diagrams in the publications. This software provides the unique possibility to make this data from the pre-digital era accessible in a reasonable amount of time. Further extensions to this package will involve the support of new diagram types (e.g. multiple lines in a single diagram column). A strong focus will lie on the documentation of the software in order to make it easier and accessible. This will involve video tutorials, more tutorials for the various diagrams directly embedded into the software, and there are still some parts of the software that are not yet sufficiently document.

In chapter \ref{chp:psyplot}, I further described the interactive visualization framework psyplot \citep{Sommer2017}, a cross-platform open source python project that combines plotting and data management into a single framework that can be used conveniently via command-line and a \gls{gui}. The software differs from most of the visual analytic software such that it focuses on extensibility and flexibility and can therefore be used in a variety of research questions. It particularly provides the basis for the straditize software (chapter \ref{chp:straditize}) and for multiple of the analysis in the chapters \ref{chp:gridding} and \ref{chp:gwgen}. psyplot currently provide visualization methods that range from simple line plots, to density plots, regression analysis and geo-referenced visualization in two dimensions. In the future, this will be enhanced with 3D visualization methods to provide the first visualization tool in climate research that can be used conveniently from the command line \citep{Sommer2019a}. 

In the second part of my thesis I described two new computational models that leverage site-based information into models for large-scale prediction of paleo environments. The first one is the pyleogrid package, a new method for a spatio-temporal gridded climate estimate from a database of proxy-climate reconstruction. It is a probabilistic extension of the method by \cite{MauriDavisCollinsEtAl2015} and \cite{DavisBrewerStevensonEtAl2003} that provides reliable uncertainties by incorporating the intrinsic dating and proxy-climate reconstruction uncertainties of the individual sites, in order to generate a product that can conveniently be used for data-model intercomparison. In addition, this framework contains two novel methods, (1) a model to predict dating uncertainties based on the age of the pollen sample and the time-difference to the closest chronological control point, and (2) a new probabilistic version of the \gls{mat}, based on constrained Gibbs sampling algorithms for the age of the samples and the climate reconstructions. The ensemble method is very scalable, both in terms of the size of the spatio-temporal domain, and the computational resources that are used for the prediction. This method will be used in the near future to generate a climate reconstruction for the entire northern hemisphere that is based on the POLNET database as described in \cite{DavisKaplan2017}. Further developments will also concentrate on a revision of the age uncertainty estimate using the recent database by \cite{WangGoringMcGuire2019} which contains standardized chronologies for more than 500 datasets.

Finally, the second model in chapter \ref{chp:gwgen} describes the new global weather generature GWGEN \citep{SommerKaplan2017b}, a statistical model for a temporal downscaling of monthly to daily climatology. This model can be applied on the entire globe whilst being parameterized based on a large dataset of weather stations with more than 50~million records. The aim is to provide a tool that can be implemented in a global vegetational model for paleo environments. These models require daily meteorology as input which poses a considerable challenge considering the long simulation period they have.

I additionally developed another model in collaboration, that is not related to paleo and therefore not included in the main part of this thesis. This model, the Integrated Urban Complexity Model (IUCM) \citep{CremadesSommer2019}, presents a new method that simulates the transformation of urban areas while focusing on a low energy consumption from urban mobility. I mention this model here because it also incorporates the infrastructural methodologies that I developed for psyplot and GWGEN. This additional use-case highlights one of the important aspects of open-source software development, that is a flexible, extensible and sustainable modular framework, where the packages related to a specific product can be used in multiple other products. Other examples for it are the \textit{docrep} and \textit{sphinx-nbexamples} packages \citep{Sommer2018a, Sommer2018c} that I primarily developed for psyplot but are used in a variety of recently developed packages now \citep[e.g.][]{AbernatheyUchidaBuseckeEtAl2017, BanihirweLongAltuntasEtAl2019, Uchida2018}.

\printbibliography[heading=subbibintoc]

\end{refsection}